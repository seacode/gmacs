\documentclass[]{article}
\usepackage{lmodern}
\usepackage{amssymb,amsmath}
\usepackage{ifxetex,ifluatex}
\usepackage{fixltx2e} % provides \textsubscript
\ifnum 0\ifxetex 1\fi\ifluatex 1\fi=0 % if pdftex
  \usepackage[T1]{fontenc}
  \usepackage[utf8]{inputenc}
\else % if luatex or xelatex
  \ifxetex
    \usepackage{mathspec}
  \else
    \usepackage{fontspec}
  \fi
  \defaultfontfeatures{Ligatures=TeX,Scale=MatchLowercase}
\fi
% use upquote if available, for straight quotes in verbatim environments
\IfFileExists{upquote.sty}{\usepackage{upquote}}{}
% use microtype if available
\IfFileExists{microtype.sty}{%
\usepackage{microtype}
\UseMicrotypeSet[protrusion]{basicmath} % disable protrusion for tt fonts
}{}
\usepackage[margin=1in]{geometry}
\usepackage{hyperref}
\hypersetup{unicode=true,
            pdftitle={A generalized size-structured assessment model for Crustaceans},
            pdfborder={0 0 0},
            breaklinks=true}
\urlstyle{same}  % don't use monospace font for urls
\usepackage{graphicx,grffile}
\makeatletter
\def\maxwidth{\ifdim\Gin@nat@width>\linewidth\linewidth\else\Gin@nat@width\fi}
\def\maxheight{\ifdim\Gin@nat@height>\textheight\textheight\else\Gin@nat@height\fi}
\makeatother
% Scale images if necessary, so that they will not overflow the page
% margins by default, and it is still possible to overwrite the defaults
% using explicit options in \includegraphics[width, height, ...]{}
\setkeys{Gin}{width=\maxwidth,height=\maxheight,keepaspectratio}
\IfFileExists{parskip.sty}{%
\usepackage{parskip}
}{% else
\setlength{\parindent}{0pt}
\setlength{\parskip}{6pt plus 2pt minus 1pt}
}
\setlength{\emergencystretch}{3em}  % prevent overfull lines
\providecommand{\tightlist}{%
  \setlength{\itemsep}{0pt}\setlength{\parskip}{0pt}}
\setcounter{secnumdepth}{0}
% Redefines (sub)paragraphs to behave more like sections
\ifx\paragraph\undefined\else
\let\oldparagraph\paragraph
\renewcommand{\paragraph}[1]{\oldparagraph{#1}\mbox{}}
\fi
\ifx\subparagraph\undefined\else
\let\oldsubparagraph\subparagraph
\renewcommand{\subparagraph}[1]{\oldsubparagraph{#1}\mbox{}}
\fi

%%% Use protect on footnotes to avoid problems with footnotes in titles
\let\rmarkdownfootnote\footnote%
\def\footnote{\protect\rmarkdownfootnote}

%%% Change title format to be more compact
\usepackage{titling}

% Create subtitle command for use in maketitle
\newcommand{\subtitle}[1]{
  \posttitle{
    \begin{center}\large#1\end{center}
    }
}

\setlength{\droptitle}{-2em}

  \title{A generalized size-structured assessment model for Crustaceans}
    \pretitle{\vspace{\droptitle}\centering\huge}
  \posttitle{\par}
    \author{}
    \preauthor{}\postauthor{}
      \predate{\centering\large\emph}
  \postdate{\par}
    \date{12:26 February 25, 2019}


\begin{document}
\maketitle

{
\setcounter{tocdepth}{2}
\tableofcontents
}
\pagenumbering{gobble}

\begin{center}\rule{0.5\linewidth}{\linethickness}\end{center}

\begin{centering}
GMACS Development Team \\
  Andre Punt, James Ianelli, Steve Martell, Cody Szuwalski  
  \\
\fontfamily{cmr}
\fontsize{10}{12}
\selectfont
February 25, 2019 \\
\end{centering}

\begin{center}\rule{0.5\linewidth}{\linethickness}\end{center}

\newcounter{saveEq} \def\putEq{\setcounter{saveEq}{\value{equation}}}
\def\getEq{\setcounter{equation}{\value{saveEq}}}
\def\tableEq{ % equations in tables
    \putEq \setcounter{equation}{0}
    \renewcommand{\theequation}{T\arabic{table}.\arabic{equation}}
    \vspace{-5mm}
    } \def\normalEq{ % renew normal equations
    \getEq
    \renewcommand{\theequation}{\arabic{section}.\arabic{equation}}}

\def\puthrule{ %thick rule lines for equation tables
    \hrule \hrule \hrule \hrule \hrule}

\def\puthrule{ %thick rule lines for equation tables
    \hrule \hrule \hrule \hrule \hrule}

\newcommand{\fspr}{$F_{\textnormal{SPR}}$}

\newcommand{\bspr}{$B_{\textnormal{SPR\%}}$}

\newcommand{\fmsy}{$F_{\textnormal{MSY}}$}
\newcommand{\bmsy}{$B_{\textnormal{MSY}}$}

\section{Basic population dynamics}\label{basic-population-dynamics}

The basic dynamics account for growth, mortality, maturity state and
shell condition (although most of the equations omit these indices for
simplicity):

\begin{align}
N_{hji} &= \left( \left( \textbf{I}-\textbf{P}_{hji-1} \right) + \textbf{X}_{hji-1} \textbf{P}_{hji-1} \right) \textbf{S}_{hji-1} N_{hji-1} + \widetilde{R}_{hji}   
\end{align}

where \(N_{hji}\) is the number of animals by size-class of sex \(h\) at
the start of season \(j\) of year \(i\), \(\textbf{P}_{hji}\) is a
matrix with diagonals given by vector of molting probabilities for
animals of sex h at the start of season \(j\) of year \(i\),
\(\textbf{S}_{hji}\) is a matrix with diagonals given by the vector of
probabilities of surviving for animals of sex \(h\) during time-step
\(j\) of year \(i\) (which may be of zero duration):

\begin{align}
S_{hjill} &= exp\left(-Z_{hjil} \right) \\ 
S_{hjill} &= 1-\frac{Z_{hjil}}{\widetilde{Z}_{hjil}  } \left(1- exp\left(-Z_{hjil} \right) \right) 
\end{align}

\(\textbf{X}_{hji}\) is the size-transition matrix (probability of
growing from one size-class to each of the other size-classes or
remaining in the same size class) for animals of sex \(h\) during season
\(j\) of year \(i\), is the recruitment (by size-class) to gear g during
season \(j\) of year \(i\) (which will be zero except for one season --
the recruitment season), is the total mortality for animals of sex \(h\)
in size- class \(l\) during season \(j\) of year \(i\), and is the
probability of encountering the gear for animals of sex \(h\) in
size-class \(l\) during season \(j\) of year \(i\). Equation A.2a
applies when mortality is continuous across a time-step and equation
A.2b applies when a time-step is instantaneous. Equation A.1a can be
modified to track old and new shell crab (under the assumption that both
old and new shell crab molt), i.e.:

\begin{align}
N^{new}_{hji} &= \textbf{X}_{hji-1}\textbf{P}_{hji-1}  \textbf{S}_{hji-1} \left(  N^{new}_{hji-1} + N^{old}_{hji-1} \right) +  \widetilde{R}_{hji}   \\
N^{old}_{hji} &= \left( \textbf{I}-\textbf{P}_{hji-1}\right)  \textbf{S}_{hji-1}\textbf{P}_{hji-1}  \left(  N^{new}_{hji-1} + N^{old}_{hji-1} \right) 
\end{align}

There are several ways to specify the initial conditions for the model
(i.e., the numbers-at- size at the start of the first year, \(i_{1}\)).

\begin{itemize}
\item
  An equilibrium size-structure based on constant recruitment and either
  no fishing for any of the fleets or (estimated or fixed) fishing
  mortality by fleet. The average recruitment is an estimated parameter
  of the model.
\item
  An individual parameter for each size- class, i.e.:
\end{itemize}

\begin{align}
 N_{hi_{1}1} = exp(\delta_{hi_{1}l})
\end{align}

\begin{itemize}
\tightlist
\item
  An overall total recruitment multiplied by offsets for each
  size-class, i.e.:
\end{itemize}

\begin{align}
 N_{hi_{1}1} = \frac {R_{init}exp(\delta_{hi_{1}l})} {\sum_{h'} \sum_{l'} {exp(\delta_{hi_{1}l'})}}
\end{align}

\subsection{B. Recruitment}\label{b.-recruitment}

Recruitment occurs once during each year. Recruitment by sex and
size-class is the product of total recruitment, the split of the total
recruitment to sex and the assignment of sex-specific recruitment to
size-classes, i.e.:

\begin{align}
 \widetilde{R}_{hjil} =  \bar{R} e^{\epsilon_{i}}
  \begin{cases}
    (1 + e^{\theta_{i}})^{-1} p_{hl} & \text{if h = males} \\[2ex]
    \theta_{i} (1 + e^{\theta_{i}})^{-1} p_{hl} & \text{if h = females} \\
    \end{cases}
    \end{align}

where \(\bar{R}\) is median recruitment, \(\theta_{i}\) determines the
sex ratio of recruitment during year \(i\), and \(p_{hl}\) is the
proportion of the recruitment (by sex) that recruits to size-class
\(l\):

\begin{align}
 p_{hil} = \int_{L_{low}}^{L_{high}} \frac {\frac{le^{-l/\beta_{h}}}{\beta_{h}}^{(\alpha^{h}/\beta^{h})-1}}{\Gamma(\alpha_{h}/\beta{h})} dl
\end{align}

where \(\alpha_{h}\) and \(\beta_{h}\) are the parameters that define a
gamma function for the distribution of recruits to size-class \(l\).
Equation X can be restricted to a subset of size-classes, in which case
the results from Equation X are normalized to sum to 1 over the selected
size-classes.

\subsection{C. Total mortality / probability of
capture}\label{c.-total-mortality-probability-of-capture}

Total mortality is the sum of fishing mortality and natural mortality,
i.e.:

\begin{align}
 Z_{hijl} = \rho_{ij}M_{hi}\tilde{M}_{l} + \sum_{f} S_{fhijl}(\lambda_{fhijl} + \Omega_{fhijl}(1-\lambda_{fhijl}))F_{fhijl}
\end{align}

where \(\rho_{ij}\) is the proportion of natural mortality that occurs
during season \(j\) for year \(i\), \(M_{hi}\) is the rate of natural
mortality for year \(i\) for animals of sex \(h\) (applies to animals
for which \(\tilde{M}_{l} = 1\)), \(\tilde{M}_{l}\) is the relative
natural mortality for size-class \(l\), \(S_{fhijl}\) is the (capture)
selectivity for animals of sex \(h\) in size- class \(l\) by fleet \(f\)
during season \(j\) of year \(i\), \(\lambda_{fhijl}\) is the
probability of retention for animals of sex \(h\) in size-class \(l\) by
fleet \(f\) during season \(j\) of year \(i\), \(\Omega_{fhijl}\) is the
mortality rate for discards of sex \(h\) in size-class \(l\) by fleet
\(f\) during season \(j\) of year \(i\), and \(F_{fhijl}\) is the
fully-selected fishing mortality for animals of sex \(h\) by fleet \(f\)
during season \(j\) of year \(i\).

The probability of capture (occurs instantaneously) is given by:

\begin{align}
 \widetilde{Z}_{hijl} = \sum_{f} S_{fhijl}F_{fhij}
\end{align}

Note that Equation C.2 is computed under the premise that fishing is
instantaneous and hence that there is no natural mortality during season
\(j\) of year \(i\). The logarithms of the fully-selected fishing
mortalities by season are modelled as:

\begin{align}
 ln(F_{fhij}) = ln(F_{fh}) + \epsilon_{fhij}  & \text{  if h = males} \\[2ex]
 ln(F_{fhij}) = ln(F_{fh}) + \theta_{f} + \epsilon_{fhij}  & \text{  if h = females} 
\end{align}

where \(F_{fh}\) is the reference fully-selected fishing mortality rate
for fleet \(f\), \(\theta_{f}\) is the offset between female and male
fully-selected fishing mortality for fleet \(f\), and
\(\epsilon_{fhij}\) are the annual deviation of fully-selected fishing
mortality for fleet \(f\) (by sex). Natural mortality can depend on time
according to several functional forms:

\begin{itemize}
\tightlist
\item
  Natural mortality changes over time as a random walk, i.e.:
\end{itemize}

\begin{align}
 M_{hi} = 
  \begin{cases}
    M_{hi_{1}} & \text{if i = $i_{1}$ } \\[2ex]
    M_{hi-1}e^{\psi_{hi}} & \text{otherwise} \\
    \end{cases}
    \end{align}

where \(M_{hi_{1}}\) is the rate of natural mortality for sex \(h\) for
the first year of the model, and \(\psi_{hi}\) is the annual change in
natural mortality.

\begin{itemize}
\tightlist
\item
  Natural mortality changes over time as a spline function. This option
  follows Equation X, except that the number of knots at which
  \(\psi_{hi}\) is estimated is specified.\\
\item
  Blocked changes. This option follows Equation C.5a, except that
  \(\psi_{hi}\) changes between `blocks' of years, during which
  \(\psi_{hi}\) is constant.\\
\item
  Blocked natural mortality (individual parameters). This option
  estimates natural mortality as parameters by block, i.e.:
\end{itemize}

\begin{align}
          M_{hi} = e^{\psi_{hi}}
\end{align}

where \(\psi_{hi}\) changes in blocks of years.

\begin{itemize}
\tightlist
\item
  Blocked offsets (relative to reference). This option captures the
  intent of the previous option, except that the parameters are relative
  to natural mortality in the first year, i.e.:
\end{itemize}

\begin{align}
          M_{hi} = M_{hi_{1}}e^{\psi_{hi}}
\end{align}

It is possible to `mirror' the values for the \(\psi_{hi}\) parameters
(between sexs and between blocks), which allows male and female natural
mortality to be the same, and for natural mortality to be the same for
discontinuous blocks (based on Equations X and X). The deviations in
natural mortality can also be penalized to avoid unrealistic changes in
natural mortality to fit `quirks' in the data.

\subsection{D. Landings, discards, total
catch}\label{d.-landings-discards-total-catch}

The model keeps track of (and can be fitted to) landings, discards,
total catch by fleet, whose computation depends on whether the fisheries
in season t are continuous or instantaneous.

Quantity Continuous mortality Instantaneous mortality Landed catch

(D.1a)

\subsection{Discards}\label{discards}

(D.1b)

\subsection{Total catch}\label{total-catch}

(D.1c)

Landings, discards, and total catches by fleet can be aggregated over
sex (e.g., when fitting to removals reported as sex-combined). Equations
D.1a -1c are extended naturally for the case in which the population is
represented by shell condition and/or maturity status (given the
assumption that fishing mortality, retention and discard mortality
depend on sex and time, but not on shell condition nor maturity status).
Landings, discards, and total catches by fleet can be reported in
numbers (Equations D.1a -- D.1c) or in terms of weight. For example, the
landings, discards, and total catches by fleet, season, year, and sex
for the total (over size-class) removals are computed as:

\begin{align}
 C^{Land}_{fhij} = \sum_{l}  C^{Land}_{fhijl}w_{hil} \\
 C^{Disc}_{fhij} = \sum_{l}  C^{Disc}_{fhijl}w_{hil} \\
 C^{Total}_{fhij} = \sum_{l}  C^{Total}_{fhijl}w_{hil} \\
\end{align}

where \(C^{Land}_{fhij}\), \(C^{Disc}_{fhij}\), and \(C^{Total}_{fhij}\)
are respectively the landings, discards, and total catches in weight by
fleet, season, year, and sex for the total (over size-class) removals,
and \(w_{hil}\) is the weight of an animal of sex h in size-class l
during year i.

\subsection{E. Selectivity / retention}\label{e.-selectivity-retention}

Many options exist related to selectivity (the probability of
encountering the gear) and retention (the probability of being landed
given being captured). The options for selectivity are:

\begin{itemize}
\tightlist
\item
  Individual parameters for each size-class (in log-space); normalized
  to a maximum of 1 over all size-classes.
\item
  Individual parameters for a subset of the size-classes (in log-space).
  Selectivity must be specified for a contiguous range of size-classes
  starting with the first size-class. Selectivity for any size-classes
  outside of the specified range is set to that for last size-class for
  which selectivity is treated as estimable.
\item
  Logistic selectivity. Two variants are available depending of the
  parametrization:
\end{itemize}

\begin{align}
   S_{l} = \frac {1} {1 + exp(\frac{ln19(\bar{L}_l - S_{50})}{S_{95}-S_{50}})} \\[2ex]
   S_{l} = \frac {1} {1 + exp(\frac{(\bar{L}_l - S_{50})}{\sigma_{S}})} 
  \end{align}

where \(S_{50}\) is the size corresponding to 50\% selectivity,
\(S_{95}\) is the size corresponding to 95\% selectivity, \(\sigma_{S}\)
is the ``standard deviation'' of the selectivity curve, and
\(\bar{L}_l\) is the midpoint of size-class l.

\begin{itemize}
\tightlist
\item
  All size-classes are equally selected.
\item
  Selectivity is zero for all size-classes.
\end{itemize}

It is possible to assume that selectivity for one fleet is the product
of two of the selectivity patterns. This option is used to model cases
in which one survey is located within the footprint of another survey.
The options to model retention are the same as those for selectivity,
except that it is possible to estimate an asymptotic parameter, which
allows discard of animals that would be ``fully retained'' according to
the standard options for (capture) selectivity. Selectivity and
retention can be defined for blocks of contiguous years. The blocks need
not be the same for selectivity and retention, and can also differ
between fleets and sexs.

\subsection{F. Growth}\label{f.-growth}

Growth is a key component of any size-structured model. It is modelled
in terms of molt probability and the size-transition matrix (the
probability of growing from each size-class to each of the other
size-classes, constrained to be zero for sizes less than the current
size). Note that the size-transition matrix has entries on its diagonal,
which represent animals that molt but do not change size-classes

\subsubsection{F.1 Molt probability}\label{f.1-molt-probability}

There are three options for modelling the probability of molting as a
function of size:

\begin{itemize}
\tightlist
\item
  Pre-specified probability
\item
  Constant probability
\item
  Logistic probability, i.e.:
\end{itemize}

\begin{align}
  P_{l,l} = \frac {1}{1-(1+exp(\frac{\bar{L}_{l}-P_{50}}{\sigma_{P}}))}
 \end{align}

where \(P_{50}\) is the size at which the probability of molting is 0.5
and \(\sigma{P}\) is the ``standard deviation'' of the molt probability
function. Molt probability is specified by sex and can change in blocks.

\subsubsection{F.2 Size-transition}\label{f.2-size-transition}

The proportion of animals in size-class \(l\) that grow to be in
size-class \(l'\) (\(X_{l,l'}\)) can either be pre-specified by the user
or determined using a parametric form (specified for one sex and one
time-blocks):

\begin{itemize}
\tightlist
\item
  The size-increment is gamma-distributed:
\end{itemize}

\begin{align}

 X_{l,l'} = \int_{L_{low}}^{L_{high}} \frac {((l-\bar{L}_{l})/\tilde{\beta})^{I_{l}/\tilde{\beta}-1} e^{-(l-\bar{L}_{l})/\tilde{\beta}}}
                                          {\Gamma( I_{l}/\tilde{\beta})} dl

 \end{align}

where is the `expected' growth increment for an animal in size-class i
(a linear function of the mid-point of size-class i), determines the
variation in growth among individuals, and and are respectively the
lower and upper bounds of size-class j.

\begin{itemize}
\tightlist
\item
  The size after increment is gamma-distributed, i.e.:
\end{itemize}

\begin{align}

 X_{l,l'} = \int_{L_{low}}^{L_{high}} \frac {(l/\tilde{\beta})^{(\bar{L}_{l} +I_{l})/\tilde{\beta}-1} e^{-(l/\tilde{\beta})}}
                                          {\Gamma((\bar{L}_{l}+I_{l})/\tilde{\beta})} dl

 \end{align}

\begin{itemize}
\tightlist
\item
  The size-increment is normally-distributed, i.e.:
\end{itemize}

\begin{align}
 X_{l,l'} = \int_{L_{low}}^{L_{high}} \frac{e^{-(l-\bar{L}_{l}-I_{l})^{2}/(2\tilde{\beta}^{2})}}
                                          {\sqrt{2\pi}\tilde{\beta}} dl
 \end{align}

\begin{itemize}
\tightlist
\item
  There is individual variation in the growth parameters and k
  (equivalent to the parameters of a linear growth increment equation
  given the assumption of von Bertlanffy growth), i.e.:
\end{itemize}

\begin{align}
  X_{l,l'} = \int_{L_{low}}^{L_{high}} \! \int_{L_{low}}^{L_{high}} \! \int_{0}^{\infty} \! \int_{0}^{\infty}
            \frac{1}{L_{hi,l}-L_{low_{l}}}
            \frac{e^{-(ln(L_{\infty})-\bar{L_{\infty}})^{2}/(2\sigma^{2}_{L_{\infty}})}}
                                          {\sqrt{2\pi}\sigma^{2}_{L_{\infty}}}
            \frac{e^{-(ln(k)-\bar{k})^{2}/(2\sigma^{2}_{k})}}
                                          {\sqrt{2\pi}\sigma^{2}_{L_{k}}}  dL_{L_{\infty}} dk dl_{l'} dl_{l}
                                          
 \end{align}

\begin{itemize}
\tightlist
\item
  There is individual variation in the growth parameter \(L_{\infty}\):
\end{itemize}

\begin{align}
  X_{l,l'} = \int_{L_{low}}^{L_{high}} \! \int_{L_{low}}^{L_{high}} \! \int_{0}^{\infty} 
            \frac{1}{L_{hi,l}-L_{low_{l}}}
            \frac{e^{-(ln(L_{\infty})-\bar{L_{\infty}})^{2}/(2\sigma^{2}_{L_{\infty}})}}
                                          {\sqrt{2\pi}\sigma^{2}_{L_{\infty}}} dL_{L_{\infty}} dl_{l'} dl_{l}
                                          
 \end{align}

\begin{itemize}
\tightlist
\item
  There is individual variation in the growth parameters k:
\end{itemize}

\begin{align}
  X_{l,l'} = \int_{L_{low}}^{L_{high}} \! \int_{L_{low}}^{L_{high}} \! \int_{0}^{\infty} 
            \frac{1}{L_{hi,l}-L_{low_{l}}}
            \frac{e^{-(ln(k)-\bar{k})^{2}/(2\sigma^{2}_{k})}}
                                          {\sqrt{2\pi}\sigma^{2}_{k}} dk dl_{l'} dl_{l}
                                          
 \end{align}

The size-transition matrix is specified by sex and can change in blocks.

Table 1.1. The symbols used to define the population dynamics model

\begin{table}
  \centering
  \caption{Mathematical notation, symbols and descriptions.}
  \label{tab:notation}
  \begin{tabular}{cl}
  \hline
  Symbol  & Description \\
  \hline
  \multicolumn{2}{l}{\underline{Index}}\\
      $g$ & group \\
      $h$ & sex \\
      $i$ & year \\
      $j$ & time step (years) \\
      $k$ & gear or fleet \\
      $l$ & index for size class \\
      $m$ & index for maturity state \\
      $o$ & index for shell condition. \\
  \multicolumn{2}{l}{\underline{Leading Model Parameters}}\\
      $M$         & Instantaneous natural mortality rate\\
      $\bar{R}$   & Average recruitment\\
      $\ddot{R}$  & Initial recruitment\\
      $\alpha_r$  & Mode of size-at-recruitment\\
      $\beta_r $  & Shape parameter for size-at-recruitment\\
      $R_0$       & Unfished average recruitment\\
      $\kappa$    & Recruitment compensation ratio\\
  \multicolumn{2}{l}{\underline{Size schedule information}}\\
      $w_{h,l}$   & Mean weight-at-size $l$ \\
      $m_{h,l}$   & Average proportion mature-at-size $l$ \\
  \multicolumn{2}{l}{\underline{Per recruit incidence functions}} \\
      $\phi_B$    & Spawning biomass per recruit \\
      $\phi_{Q_k}$& Yield per recruit for fishery $k$\\
      $\phi_{Y_k}$& Retained catch per recruit for fishery $k$ \\
      $\phi_{D_k}$& Discarded catch per recruit for fishery $k$ \\
  \multicolumn{2}{l}{\underline{Selectivity parameters}} \\
      $a_{h,k,l}$ & Size at 50\% selectivity in size interval $l$\\
      $\sigma_{s_{h,k}}$ & Standard deviation in size-at-selectivity\\
      $r_{h,k,l}$ & Size at 50\% retention\\
      $\sigma_{y_{h,k}}$ & Standard deviation in size-at-retention\\
      $\xi_{h,k}$ & Discard mortality rate for gear $k$ and sex $h$\\
  \hline
  \end{tabular}
\end{table}

\begin{align}
    & \mbox{\underline{model parameters}} \nonumber \\
    & \Theta = (M_h,\bar{R},\ddot{R},\alpha_r,\beta_r,R_0,\kappa) \label{T1.1} \\
    & M_h > 0 , \bar{R} > 0, \ddot{R}>0, \alpha_r > 0, \beta_r > 0, R_0>0,\kappa > 1.0 \label{T1.2} \\
    & \Phi = (\alpha_h,\beta_h,\varphi_h) \label{T1.3} \\
    & \mbox{\underline{size-schedule information}} \nonumber \\
    & \vec{l},\vec{x} \quad \mbox{vector of size intervals and midpoints, respectively} \nonumber\\
    & a_{h,l} = (\alpha_h + \beta_h l)/\varphi_h \label{T1.4} \\
    & p({l},{l'})_h =\textbf{G}_h= \int_{l}^{l+\Delta l} \frac{ l^{(a_{h,l}-1)} \exp(l/\varphi_h) } { \Gamma(a_{h,l}) l^{(a_{h,l})} } dl \label{T1.5} \\
    & \mbox{\underline{recruitment size-distribution}} \nonumber \\
    & \alpha = \alpha_r / \beta_r  \\
    & p[\textbf{r}] = \int_{x_l-0.5\Delta x}^{x_l+0.5\Delta x} \frac{x^{(\alpha-1)}\exp(- x / \beta_r)}{\Gamma(\alpha)\beta_r^\alpha}dx \label{T1.7}\\
    & \textbf{r}_h = 0.5 p[\textbf{r}] \ddot{R} \label{T1.8}\\
    & \mbox{\underline{growth and survival}} \nonumber \\
    & \textbf{A}_h = \textbf{G}_h [\exp(-M_h) (\textbf{I}_n)_{l,l'}]\label{T1.10}\\
    & \textbf{B}_h = \textbf{G}_h [\exp(-M_h - \textbf{f}_{h,l}) (\textbf{I}_n)_{l,l'}] \label{T1.12}\\
    & \mbox{\underline{survivorship to size}} \nonumber \\
    & \textbf{u}_h   = -(\textbf{A}_h - (\textbf{I}_n)_{l,l'})^{-1} (p[\textbf{r}]) \label{T1.13a}\\
    & \textbf{v}_h   = -(\textbf{B}_h - (\textbf{I}_n)_{l,l'})^{-1} (p[\textbf{r}]) \label{T1.13b}\\
    & \mbox{\underline{steady-state conditions}}\nonumber \\
    & B_0 = R_0 \sum_h \lambda_h \sum_l \textbf{u}_{h,l} w_{h,l} m_{h,l} \label{T1.14} \\
    & \tilde{B} = \tilde{R}\sum_h \lambda_h \sum_l \textbf{v}_{h,l} w_{h,l} m_{h,l} \label{T1.16} \\
    & \mbox{\underline{stock-recruitment parameters}}\nonumber \\
    & s_o = \kappa R_0 / B_0 \label{T1.17} \\
    & \beta = (\kappa -1)/B_0 \label{T1.18} \\
    & \tilde{R} = \frac{s_o \tilde{\phi}_B -1}{\beta \tilde{\phi}_B} \label{T1.19} \\
\end{align}


\end{document}
